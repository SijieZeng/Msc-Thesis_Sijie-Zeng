\chapter*{Preface}
\addcontentsline{toc}{chapter}{Preface}


Undertaking this master's thesis has been both an intellectually enriching and personally transformative journey. I entered the world of Particle Filter-Based Signalized Intersection Queue Estimation (PF-SIQE) without prior knowledge of particle filters, but I felt a strong connection with my supervisor, Andreas, from our very first meeting. His enthusiasm for the topic and supportive approach made me feel comfortable venturing into a new and unfamiliar field. I treasure this opportunity, not only for the academic insights gained but also for the chance to work under such a dedicated and inspiring mentor.

Though I have always enjoyed research, I quickly realized that managing a complex project like this one required skills I had not yet fully developed. Time management, project planning, risk control, and the ability to respond to unexpected challenges were constant tests of my resolve. These elements, I learned, are as critical as the research itself. I deeply appreciate Andreas for his guidance, particularly his insistence on three principles: being honest with myself and others, constantly challenging myself, and addressing feedback thoroughly. These principles have become cornerstones of my approach to research and will undoubtedly continue to shape my future work.

Reflecting on the journey, it was the process of pushing through discomfort and stepping outside my comfort zone that fostered the most personal growth. Balancing what I wanted to do with what I had to do taught me invaluable lessons about perseverance and adaptability. This growth was not confined to academic learning; it extended to self-awareness. I now understand that life, like research, is a balance between exploration of the external world and introspection. It’s this combination that allows for meaningful progress.

My research in transportation and planning has spanned eight years, initially driven by a sense of social responsibility. Over time, my curiosity about human nature grew, leading me to focus on human-oriented research—particularly how humans make decisions and process information. The novel decision-making model developed in this thesis is a key step toward my future research goals in understanding human cognition, especially in the context of driving. Moreover, working with the particle filter algorithm has equipped me with new tools and perspectives, offering a glimpse into how probability theory can explain complex real-world phenomena and providing a structured, modular framework for future problem-solving.

Throughout this process, I have received invaluable support from many people. I am deeply grateful to Andreas for his mentorship and guidance. His ability to adapt his supervisory style to suit my needs as a student has been crucial to my success. I would also like to thank the members of my assessment committee, Henk and Yufei, for their thoughtful feedback and encouragement. I am especially grateful to my study advisor, Jorieke, who supported me not only in the academic aspects but also helped me navigate personal challenges and project management. Jing, my psychologist, provided me with a reflective space to process my thoughts and emotions, enabling me to grow as both a researcher and a person. And of course, I must mention my cat, Du-du, whose quiet companionship often made me feel as though he was my true supervisor during long hours of work. Last but not least, I want to thank my family, friends, and my boyfriend for their unwavering support throughout this journey.

The technical goals of my research—building a predictive framework, conducting sensitivity analysis, and running simulations—proved more ambitious than I initially anticipated. Although the project is not perfect, it has laid a strong foundation. I developed a novel traffic light decision model and made significant strides in understanding measurement noise distributions from loop detectors. While some results were obtained and validated, there is still room for further exploration. Even as this chapter closes, I remain deeply interested in refining this work. I look forward to continuing this research and seeing it fully realized in the future.


\begin{flushright}
{\makeatletter\itshape
    \@author \\
    Delft, \monthname{} \the\year{}
\makeatother}
\end{flushright}
