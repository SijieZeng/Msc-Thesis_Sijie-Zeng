\chapter{Discussion, Conclusion, and Recommendation}
\label{chapter: Discussion, Conclusion, and Recommendation}


This chapter presents a comprehensive summary of the key findings from the research and how they align with the initial objectives. The chapter also revisits the main hypotheses and assumptions made during the study, examining their theoretical and practical implications. The discussion further evaluates the significance of the results within the broader context of intelligent transportation systems (ITS) and identifies any limitations encountered. Finally, the chapter offers recommendations for future research, highlighting areas where the PF-SIQE model can be further refined and extended. Through these insights, the conclusion encapsulates the core contributions of this research while outlining the potential path forward.

\section{Outcomes}

This thesis set out to develop a Particle Filter-based Signalized Intersection Queue Estimator (PF-SIQE) to estimate real-time queue lengths at signalized intersections, addressing several research objectives. Below is a summary of the key outcomes related to these objectives.

\begin{enumerate}
    \item \textbf{Developing a PF-SIQE architecture and algorithm framework:} \\
    The PF-SIQE architecture and algorithm framework were successfully developed, as detailed in Chapter \ref{chapter: Development of a Particle Filter-Based Signalized Intersection Queue Estimator (PF-SIQE)}. The architecture consists of three core modules: the particle filter module, the state transition function module, and the measurement function module. The modular design allows for flexibility and adaptability, making it possible to integrate various microscopic traffic flow models and observational data from different detection technologies. This meets the first research objective, as the framework is both scalable and adaptable to diverse traffic conditions and sensor inputs.

    \item \textbf{Studying detection technologies, their outputs, and corresponding noise distributions:} \\
    Chapter \ref{chapter:Literature Review} explored various detection technologies and their output measurements. The focus was on loop detectors and GPS data, with noise characteristics for these technologies discussed in Chapter \ref{chapter: Development of a Particle Filter-Based Signalized Intersection Queue Estimator (PF-SIQE)}. A key contribution of this thesis is the novel mathematical representation of noise characteristics for loop detectors, which enhances the adaptability of the PF-SIQE. While not all detection technologies were covered, the research successfully addressed this objective by examining the most relevant technologies for signalized intersection queue estimation.

    \item \textbf{Building a modular mathematical model and implementing it in simulation:} \\
    The modular mathematical model of the PF-SIQE was developed and implemented in a simulation environment, as described in Chapter \ref{chapter: Development of a Particle Filter-Based Signalized Intersection Queue Estimator (PF-SIQE)}. This model integrates the three core modules, utilizing state transition functions and measurements to estimate vehicle queues at intersections. The implementation verified the model's ability to function as designed, thus achieving the third objective.

    \item \textbf{Validation of the PF-SIQE:} \\
    Initial validation efforts focused on the traffic light decision-making model, demonstrating that vehicle behavior in simulations aligned with the decision strategies outlined in the model. Though additional sensitivity tests were initially planned, they were not fully completed due to time constraints. However, the validation results provide a solid foundation for the model's reliability. Further validation is recommended for future work to confirm the estimator’s performance under a broader range of conditions.
\end{enumerate}

By addressing these key objectives, this research has made significant progress in the development of a scalable, adaptable, and real-time signalized intersection queue estimator. While some areas, such as sensitivity analysis, will require future research, the outcomes presented here reflect a strong foundation for further development and practical application of the PF-SIQE.


\section{Discussions}


\subsection{Hypothesis}
This research was driven by the hypothesis that a particle filter-based algorithm can accurately estimate real-time queue lengths at signalized intersections, while offering flexibility and adaptability to different data sources. The validation of the traffic light decision-making model supported this hypothesis, demonstrating that vehicle behavior—such as the decision to stop or proceed—was consistent with the designed strategies. This provides a strong foundation for the application of particle filters in dynamic, real-time traffic scenarios. Furthermore, the modularity of the PF-SIQE framework enables it to handle various detection technologies and adapt to different traffic environments, enhancing its potential for practical implementation.

\subsection{Assumptions}
\begin{itemize}
    \item \textbf{Driver Decision Distance ($D_h = 150$ m):} \\
    It is assumed that when a vehicle is within 150 meters of the traffic signal, the driver begins making decisions to stop or proceed based on the signal status. This assumption is based on standard visibility requirements in ideal conditions. However, factors such as weather, road layout (e.g., curves, gradients), and obstacles can influence a driver's decision-making process. Relaxing this assumption would require dynamic modeling of visibility conditions and driver response times, which could alter the estimated queue lengths. In practice, adjusting the decision distance based on site-specific investigations would provide more accurate estimates for different intersections.

    \item \textbf{Vehicle Arrival and Departure:} \\
    This research assumes that the number of vehicles in the study area remains constant over time—when one vehicle exits, another immediately enters. While this simplifies the modeling process, it does not reflect real-world traffic dynamics, where vehicle arrivals and departures are more complex and stochastic. To model varying numbers of vehicles, changes would be needed in the state transition functions, where variable entry and exit rates could be introduced. The model would also need to account for factors such as vehicle platooning, traffic signal timing, and varying traffic demand across time periods.

    \item \textbf{GPS Data Labeling:} \\
    The model assumes that it is possible to identify which vehicle is associated with each GPS reading. In reality, GPS data may not always be labeled for individual vehicles, particularly in non-commercial settings. However, certain commercial fleets do share their real-time positions. Relaxing this assumption would require developing techniques to associate GPS signals with specific vehicles, such as vehicle re-identification algorithms or data fusion methods. Overlapping signals from nearby vehicles and GPS inaccuracies, particularly at low speeds, remain significant challenges to address in practical implementations.

    \item \textbf{Queue Definition:} \\
    The definition of a queue in this thesis is based on a velocity threshold: vehicles are considered to be in a queue if their velocity is below a certain threshold and they occupy a cumulative distance corresponding to the standard vehicle length. This assumption simplifies the process but has limitations. For instance, it does not account for different vehicle types, which may have varying lengths and behaviors. Additionally, determining the velocity threshold is a topic for further study, as this can significantly influence queue length estimation. A known drawback is that once a red light turns green, the queue may appear to suddenly disappear within a single time step, even though vehicles are still present in the intersection. This highlights the importance of integrating vehicle arrival and departure dynamics into future models.

    Moreover, the definition of a queue varies across the literature depending on the study's objective. For the purpose of studying signal cycle performance, defining the queue as vehicles with near standstill velocity may be most appropriate. However, other definitions could serve different purposes, such as analyzing traffic congestion or vehicle delays. As such, the choice of queue definition in this thesis is pragmatic and aligns with the objective of improving real-time traffic signal control.

    \item \textbf{Speed Loop Detector:} \\
    The algorithm used with the speed loop detector differentiates between whether a vehicle is passing through or not. However, the current model lacks consideration for situations where a vehicle is stationary on the detector. Accurately identifying when a vehicle is at a standstill on the detector is crucial for detecting and analyzing queues. This limitation needs to be addressed in future work to improve the accuracy of queue estimation and traffic flow analysis.

    \item \textbf{Comparison with Other Queue Estimators:} \\
    This research does not include a comparison with other queue estimators, such as the Extended Kalman Filter, for two main reasons: time constraints and uncertainty about the comparability of different approaches to handle noise. Kalman filters and their variants (such as the Unscented Kalman Filter, Ensemble Kalman Filter, and Adaptive Kalman Filter) approach noise and uncertainty differently than particle filters, which may make direct comparisons challenging. A comprehensive study of these variants would also require significant additional time. Therefore, such comparisons are considered potential future work.
\end{itemize}

\subsection{Theoretical Implications}
Theoretically, this research contributes to the growing body of knowledge on real-time, microscopic level traffic modeling and estimation using particle filters. The PF-SIQE framework pushes the boundaries of queue length estimation by offering a flexible, scalable architecture that can be applied across different detection technologies and traffic scenarios. This ability to adapt to diverse data sources, combined with high-resolution queue tracking, represents a significant improvement over traditional macroscopic models. Additionally, relaxing some of the assumptions mentioned earlier could lead to more realistic models, further enhancing the theoretical foundation for real-time traffic estimation.

These advancements address the immediate need for more granular traffic management solutions and open new pathways for future research in particle filter-based queue estimation algorithms. The exploration of more complex vehicle dynamics, such as varying arrival and departure rates, is a crucial area for extending the current model.

\subsection{Practical Implications}
From a practical perspective, the PF-SIQE model holds great potential for applications in intelligent transportation systems (ITS), such as Vehicle-to-Infrastructure (V2I) and Infrastructure-to-Vehicle (I2V) communications, adaptive traffic signals, and Green Light Optimal Speed Advisory (GLOSA) systems. However, several key assumptions—such as the reliance on simulated data for validation—highlight the gap between this research and real-world deployment.

Before the model can be effectively implemented in urban environments, it is necessary to validate it using real-world data. This validation would involve comparing the PF-SIQE’s estimations with actual traffic data collected from signalized intersections. One potential source of such data is phased array radar systems (refer to Prof. Hao Yue's lab in Beijing Jiaotong University), which can provide position, speed, and other detailed metrics for each vehicle at an intersection. Although these radar systems offer continuous monitoring, the data they produce may be imprecise and contain noise, necessitating extensive pre-processing before it can be used for validation purposes. Noise reduction and data cleaning will be critical steps in ensuring that the input data aligns with the assumptions of the PF-SIQE model.

In addition, real-world data collection from GPS-equipped vehicles, especially those within commercial fleets, would help validate the model's ability to estimate vehicle queues in diverse traffic conditions. While labeled GPS data may be available for certain vehicles, general traffic scenarios would require advanced vehicle identification techniques, such as data fusion from multiple sensors, to reliably track vehicles through an intersection.

Other key considerations include optimizing the placement of loop detectors and understanding the penetration rates of floating car sensors to ensure accurate data inputs for the model. Accurate real-time data on vehicle positions, speeds, and movements are essential for validating the particle filter’s state transition and measurement functions.

These refinements will help bridge the gap between the theoretical framework presented in this research and its practical deployment in real-world traffic management solutions.



\section{Recommendations for Future Work}

Several promising directions for future work arise from this research:

\subsection{State Transition Function Module}
    \begin{itemize}
        \item Integrate lateral microscopic models, such as lane change models, into the state transition function module to handle lateral vehicle movements. For instance, introducing $i_{k+1}^\text{lane} = g(x_k, i_k^{\text{lane}}, \text{lane change external factors}, n_k^\text{lateral})$ to describe lane changes and lateral motion more accurately.
        \item Simulate multiple lanes and turning motions at more complex intersections to extend the applicability of the model beyond simple intersections. These simulations would test the impact of lane changes, turning motions, and lane-specific configurations on queue estimation and traffic flow.
        \item The road layout will be expanded in the estimator to include the following characteristics:
        \begin{itemize}
            \item Number of lanes,
            \item Type of lanes (e.g., through lanes, dedicated turning lanes, bus lanes),
            \item Direction of traffic flows.
        \end{itemize}
    \end{itemize}

\subsection{Measurement Function Module}
    \begin{itemize}
        \item Incorporate data from a wider range of detection technologies, and explore different combinations of these data sources (e.g., loop detectors, GPS, phased array radar) to assess their impact on the estimator’s accuracy. 
        \item Study how the placement of loop detectors and radar affects estimation results, and optimize detector configurations for various intersection types. Phased array radar data, despite containing noise and inaccuracies, could be processed to provide detailed information on vehicle positions and speeds.
        \item Analyze how varying levels of floating sensor penetration (e.g., GPS-equipped vehicles) influence the estimator’s performance, particularly in real-world scenarios.
    \end{itemize}

\subsection{Particle Filter Module}
    \begin{itemize}
        \item Investigate how the number of particles affects estimation accuracy and computational efficiency, and test different resampling methods (e.g., systematic resampling, stratified resampling) to improve performance. 
        \item Explore alternative paradigms for particle representation, such as assigning multiple particles to individual vehicles or employing a hierarchical filtering approach where each vehicle is tracked using its own particle filter, allowing for collaborative filtering between vehicles.
    \end{itemize}

\subsection{Queue Estimation}
    Queue estimation, a key feature of the PF-SIQE, can be enhanced by refining the velocity threshold method used for determining the number of stationary vehicles. In this thesis, the queue is estimated by counting vehicles with speeds below a threshold. Future work should focus on:
    \begin{itemize}
        \item Developing more dynamic thresholding techniques that account for acceleration and deceleration during signal transitions, avoiding abrupt changes in queue estimates when vehicles start moving simultaneously at green lights.
        \item Extending the state space representation to handle varying arrival and departure rates, which would model real-world scenarios more effectively. This could involve modifying the state transition equations to include time-varying vehicle entries and exits.
    \end{itemize}

\subsection{Simulation Sensitivity Analysis}
\begin{itemize}
    \item Conduct sensitivity analyses on factors such as acceleration noise, loop detector placement, GPS noise distribution, and floating sensor penetration to better understand their influence on the estimator’s performance. These analyses will provide insights into how different detection technologies and environmental conditions affect the PF-SIQE’s accuracy.
\end{itemize}

\subsection{Real-World Implementation}
\begin{itemize}
    \item Real-world testing is a crucial next step to ensure the PF-SIQE model’s practical applicability. This testing should be conducted using actual traffic data collected from signalized intersections. Phased array radar systems, like those available at Prof. Hao Yue's lab in Beijing Jiaotong University, or GPS data from vehicles, could provide valuable insights into vehicle dynamics at intersections. These data sources, while continuous, will require careful pre-processing—such as noise filtering and data smoothing—to align with the assumptions of the PF-SIQE.
    
    \item In addition to validating the model's performance with real-world data, these tests should focus on assessing the PF-SIQE’s robustness across varying intersection types, lane configurations, and traffic conditions. This includes examining different traffic flow patterns, vehicle densities, and detection technologies. Such testing will not only fine-tune the model for practical use but also highlight areas for further refinement or adaptation of the algorithm.
\end{itemize}














