\chapter{source code}
%\label{chapter:title}
\emph{Adding source code to your report/thesis is supported with the package {\normalfont\texttt{listings}}. An example can be found below. Files can be added using {\normalfont\texttt{\textbackslash lstinputlisting[language=<language>]\{<filename>\}}}.}

\begin{lstlisting}[language=Python]
"""
ISA Calculator: import the function, specify the height and it will return a
list in the following format: [Temperature,Density,Pressure,Speed of Sound].
Note that there is no check to see if the maximum altitude is reached.
"""

import math
g0 = 9.80665
R = 287.0
layer1 = [0, 288.15, 101325.0]
alt = [0,11000,20000,32000,47000,51000,71000,86000]
a = [-.0065,0,.0010,.0028,0,-.0028,-.0020]

def atmosphere(h):
    for i in range(0,len(alt)-1):
        if h >= alt[i]:
            layer0 = layer1[:]
            layer1[0] = min(h,alt[i+1])
            if a[i] != 0:
                layer1[1] = layer0[1] + a[i]*(layer1[0]-layer0[0])
                layer1[2] = layer0[2] * (layer1[1]/layer0[1])**(-g0/(a[i]*R))
            else:
                layer1[2] = layer0[2]*math.exp((-g0/(R*layer1[1]))*(layer1[0]-layer0[0]))
    return [layer1[1],layer1[2]/(R*layer1[1]),layer1[2],math.sqrt(1.4*R*layer1[1])]
\end{lstlisting}






\emph{If a task division is required, a simple template can be found below for convenience. Feel free to use, adapt or completely remove.}

\begin{table}[htb]
    \setlength\extrarowheight{4pt}
    \centering
    \caption{Distribution of the workload}
    \label{tab:taskdivision}
    \begin{tabularx}{\textwidth}{lXX}
        \toprule
        & Task & Student Name(s) \\
        \midrule
        & Summary & \\
        Chapter 1 & Introduction &  \\
        Chapter 2 &  & \\
        Chapter 3 &  & \\
        Chapter * &  & \\
        Chapter * & Conclusion &  \\
        \midrule
        & Editors & \\
        & CAD and Figures & \\
        & Document Design and Layout & \\
        \bottomrule
    \end{tabularx}
\end{table}
